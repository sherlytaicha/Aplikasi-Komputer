\documentclass{article}

\usepackage{eumat}

\begin{document}
\begin{eulernotebook}
\eulerheading{SUBTOPIK 6 BAGIAN 2 | MENGGAMBAR GRAFIK STATISTIKA}
\begin{eulercomment}
Nama : Sherlyta Icha Nadiastuty\\
NIM : 22305141013\\
Kelas : Matematika B 2022

\begin{eulercomment}
\eulerheading{Diagram titik}
\begin{eulercomment}
Diagram titik atau disebut juga sebagai scatter plot, adalah jenis
diagram statistik yang menggunakan titik-titik untuk merepresentasikan
nilai dari dua variabel yang berbeda. Setiap titik dalam diagram
pencar mewakili satu pengamatan atau data dengan nilai-nilai yang
sesuai untuk kedua variabel tersebut.

Scatter plot sangat berguna untuk menemukan pola atau hubungan antara
dua variabel, serta untuk mengevaluasi distribusi data.

Dalam scatter plot, sumbu horizontal umumnya digunakan untuk variabel
independen (bebas), sedangkan sumbu vertikal digunakan untuk variabel
dependen (bergantung). Dengan mengamati pola penyebaran titik-titik,
kita mendapatkan wawasan tentang apakah ada korelasi antara dua
variabel dan jenis korelasi apa yang mungkin ada (positif, negatif,
atau tidak ada korelasi).

\end{eulercomment}
\eulersubheading{Berikut contoh menggambar diagram titik di EMT}
\begin{eulercomment}
Contoh 1\\
Pada contoh ini, kita akan menggambar diagram titik dengan menggunakan
fungsi plot2d().
\end{eulercomment}
\begin{eulerprompt}
>x=normal(1,100); 
>plot2d(x,x+rotright(x),>points,style=".."):
\end{eulerprompt}
\eulerimg{27}{images/EMTStatistika_SUB6_22305141013_Sherlyta Icha-001.png}
\begin{eulercomment}
Terdapat banyak style titik yang dapat kita dunakan, yaitu:\\
"[]", "\textless{}\textgreater{}", ".", "..", "...","*", "+", "\textbar{}", "-", "o",\\
"[]#", "\textless{}\textgreater{}#", "o#" (bentuk terisi)\\
"[]w", "\textless{}\textgreater{}w", "ow" (tidak transparan)

Selanjutnya, akan kita coba gambarkan diagram tersebut menggunakan
fungsi statplot(), pilih plottype="p" (karena kita akan menggambar
plot titik).
\end{eulercomment}
\begin{eulerprompt}
>statplot(x,x+rotright(x),plottype="p",pstyle=".."):
\end{eulerprompt}
\eulerimg{27}{images/EMTStatistika_SUB6_22305141013_Sherlyta Icha-002.png}
\begin{eulercomment}
Contoh 2\\
Pada contoh ini, akan kita gambarkan diagram titik menggunakan fungsi
scatterplot().
\end{eulercomment}
\begin{eulerprompt}
>\{MS,hd\}:=readtable("table1.dat",tok2:=["m","f"]);
>writetable(MS,labc=hd,tok2:=["m","f"]);
\end{eulerprompt}
\begin{euleroutput}
      Person       Sex       Age    Mother    Father  Siblings
           1         m        29        58        61         1
           2         f        26        53        54         2
           3         m        24        49        55         1
           4         f        25        56        63         3
           5         f        25        49        53         0
           6         f        23        55        55         2
           7         m        23        48        54         2
           8         m        27        56        58         1
           9         m        25        57        59         1
          10         m        24        50        54         1
          11         f        26        61        65         1
          12         m        24        50        52         1
          13         m        29        54        56         1
          14         m        28        48        51         2
          15         f        23        52        52         1
          16         m        24        45        57         1
          17         f        24        59        63         0
          18         f        23        52        55         1
          19         m        24        54        61         2
          20         f        23        54        55         1
\end{euleroutput}
\begin{eulerprompt}
>scatterplots(tablecol(MS,3:5),hd[3:5]):
\end{eulerprompt}
\eulerimg{27}{images/EMTStatistika_SUB6_22305141013_Sherlyta Icha-003.png}
\eulerheading{Diagram Garis}
\begin{eulercomment}
Diagram garis adalah penyajian data yang digunakan untuk menggambarkan
suatu keadaan berupa data berkala atau berkelanjutan.

Selain itu, diagram ini juga bisa dikatakan berhubungan dengan kurun
waktu dan untuk menunjukkan perkembangan suatu keadaan.

Diagram ini sangat tepat untuk menyajikan data untuk mengetahui
kecenderungan kelakuan atau tren, seperti produksi minyak tiap tahun,
jumlah kelahiran tiap tahun, jumlah produksi tiap jam, dan lain-lain.

Dalam diagram garis, terdapat sumbu vertikal (sumbu y) untuk
menunjukkan frekuensi dan sumbu horizontal (sumbu x) untuk menunjukkan
variabel tertentu.

\end{eulercomment}
\eulersubheading{Berikut contoh menggambar diagram garis di EMT}
\begin{eulercomment}
Contoh 1\\
Akan digambarkan diagram garis data banyaknya pelanggan di toko A
tahun 2015-2023.

Kita deskripsikan terlebih dahulu matriks x dan y,kemudian akan kita
buat tabel datanya.
\end{eulercomment}
\begin{eulerprompt}
>x=[2015,2016,2017,2018,2019,2020,2021,2022,2023]; y=[600,500,900,1000,800,850,900,1000,1200];
>writetable(x'|y',labc=["Tahun","Banyak Pelanggan"])
\end{eulerprompt}
\begin{euleroutput}
       Tahun Banyak Pelanggan
        2015              600
        2016              500
        2017              900
        2018             1000
        2019              800
        2020              850
        2021              900
        2022             1000
        2023             1200
\end{euleroutput}
\begin{eulercomment}
Selanjutnya, akan digambarkan diagram garis dengan menggunakan fungsi
statplot, dengan format:

statplot (x, y, plottype="l", lstyle="-", xl="", yl="", color=none,
vertical=0)

x : data untuk sumbu x\\
y : data untuk sumbu y\\
plotstyle : "l" (kita pilih style "l" karena berupa plot garis)\\
lstyle : style garis\\
xl : label sumbu x\\
yl : label sumbu y\\
color : warna garis\\
vertikal : vertikal

Style garis:\\
"-", "--", "-.", ".", ".-.", "-.-", "-\textgreater{}"
\end{eulercomment}
\begin{eulerprompt}
>statplot(x,y,plottype="l",lstyle="-",xl="Tahun",yl="Banyak Pelanggan",vertical=50):
\end{eulerprompt}
\eulerimg{27}{images/EMTStatistika_SUB6_22305141013_Sherlyta Icha-004.png}
\begin{eulercomment}
Kita juga bisa menambahkan judul pada grafik dengan menggunakan fungsi
title().
\end{eulercomment}
\begin{eulerprompt}
>title("Grafik Banyaknya Pelanggan Toko A Tahun 2015-2023"):
\end{eulerprompt}
\eulerimg{27}{images/EMTStatistika_SUB6_22305141013_Sherlyta Icha-005.png}
\begin{eulercomment}
Menyajikan data dalam bentuk diagram dapat memudahkan pembaca untuk
memahami data yang disajikan.

Dari diagram garis diatas, dapat kita peroleh informasi bahwa:\\
Banyaknya pelanggan di toko A tidak tetap (naik-turun)untuk setiap
tahunnya.\\
Banyaknya pelanggan paling sedikit pada tahun 2016 yaitu sebanyak 500
pelanggan, sedangkan banyak pelanggan paling banyak pada tahun 2023
yaitu sebanyak 1200 pelanggan.

Selain menggunakan fungsi statplot, kita juga dapat menggambar diagram
garis menggunakan fungsi plot2d() seperti yang sudah pernah kita
pelajari sebelumnya, yaitu sebagai berikut.
\end{eulercomment}
\begin{eulerprompt}
>plot2d(x,y,a=2015,b=2023,c=500,d=1500,style="_",xl="Tahun",yl="Banyak Pelanggan",vertical=50); title("Grafik Banyak Pelanggan Toko A Tahun 2015-2023"):
\end{eulerprompt}
\eulerimg{27}{images/EMTStatistika_SUB6_22305141013_Sherlyta Icha-006.png}
\begin{eulercomment}
a dan b : batas untuk sumbu x\\
c dan d : batas untuk sumbu y\\
style : gaya garis\\
xl : label untuk sumbu x\\
yl : label untuk sumbu y
\end{eulercomment}
\begin{eulercomment}
Selain kita dapat menggambarkan diagram garis saja atau diagram titik
saja, kita juga dapat menggambarkan diagram keduanya.

Contoh 2\\
Akan kita gambar diagram titik dan garis data hasil pemilu Jerman dari
tahun 1990 sampai 2013, diukur dalam kursi.
\end{eulercomment}
\begin{eulerprompt}
>BW := [ ...
>1990,662,319,239,79,8,17; ...
>1994,672,294,252,47,49,30; ...
>1998,669,245,298,43,47,36; ...
>2002,603,248,251,47,55,2; ...
>2005,614,226,222,61,51,54; ...
>2009,622,239,146,93,68,76; ...
>2013,631,311,193,0,63,64];
>P:=["CDU/CSU","SPD","FDP","Gr","Li"]; 
>BT:=BW[,3:7]; BT:=BT/sum(BT); YT:=BW[,1]';
>writetable(BT*100,wc=6,dc=0,>fixed,labc=P,labr=YT)
\end{eulerprompt}
\begin{euleroutput}
         CDU/CSU   SPD   FDP    Gr    Li
    1990      48    36    12     1     3
    1994      44    38     7     7     4
    1998      37    45     6     7     5
    2002      41    42     8     9     0
    2005      37    36    10     8     9
    2009      38    23    15    11    12
    2013      49    31     0    10    10
\end{euleroutput}
\begin{eulerprompt}
>BT1:=(BT.[1;1;0;0;0])'*100
\end{eulerprompt}
\begin{euleroutput}
  [84.29,  81.25,  81.1659,  82.7529,  72.9642,  61.8971,  79.8732]
\end{euleroutput}
\begin{eulercomment}
Akan kita gambarkan plot statistik sederhana, yaitu plot titik dan
garis secara bersamaan dengan menggunakan fungsi statplot dan pilih
plottype="b".
\end{eulercomment}
\begin{eulerprompt}
>statplot(YT,BT1,"b"):
\end{eulerprompt}
\eulerimg{27}{images/EMTStatistika_SUB6_22305141013_Sherlyta Icha-007.png}
\begin{eulerprompt}
>CP:=[rgb(0.5,0.5,0.5),red,yellow,green,rgb(0.8,0,0)];
\end{eulerprompt}
\begin{eulercomment}
Untuk menggabungkan deretan data statistik dalam satu plot, dapat kita
digunakan fungsi dataplot().
\end{eulercomment}
\begin{eulerprompt}
>J:=BW[,1]'; DP:=BW[,3:7]'; ...
>dataplot(YT,BT',color=CP);  ...
>labelbox(P,colors=CP,styles="[]",>points,w=0.2,x=0.3,y=0.4):
\end{eulerprompt}
\eulerimg{27}{images/EMTStatistika_SUB6_22305141013_Sherlyta Icha-008.png}
\begin{eulercomment}
Contoh 3\\
Akan digambarkan diagram titik dan garis dari data hasil panen padi
pada tahun 2016 sampai 2020.
\end{eulercomment}
\begin{eulerprompt}
>T=[2016,2017,2018,2019,2020,2021]; P=[70,50,40,30,50,60];
>writetable(T'|P',labc=["Tahun","Hasil Panen Padi (Ton)"])
\end{eulerprompt}
\begin{euleroutput}
       Tahun Hasil Panen Padi (Ton)
        2016                     70
        2017                     50
        2018                     40
        2019                     30
        2020                     50
        2021                     60
\end{euleroutput}
\begin{eulerprompt}
>statplot(T,P,"b",pstyle="o#",lstyle="-",xl="Tahun",yl="Hasil Panen Padi (Ton)",vertical=50);
>title("Grafik Hasil Panen Padi Tahun 2016-2020"):
\end{eulerprompt}
\eulerimg{27}{images/EMTStatistika_SUB6_22305141013_Sherlyta Icha-009.png}
\begin{eulercomment}
Kita juga menggambar diagram titik dan garis secara bersama dengan
menggunakan fungsi plot2d(), yaitu sebagai berikut.
\end{eulercomment}
\begin{eulerprompt}
>plot2d(T,P,2016,2020); plot2d(T,P,>points,style="o#",>add); title("Grafik Hasil Panen Padi Tahun 2016-2020"):
\end{eulerprompt}
\eulerimg{27}{images/EMTStatistika_SUB6_22305141013_Sherlyta Icha-010.png}
\begin{eulercomment}
Dari grafik di atas, dapat dengan mudah kita ketahui bahwa panen padi
paling banyak yaitu pada tahun 2016 (70 ton) dan paling sedikit yaitu
pada tahun 2019 (30 ton).

\begin{eulercomment}
\eulerheading{Kurva Regresi}
\begin{eulercomment}
Kurva regresi adalah representasi grafis dari model regresi yang
digunakan untuk memodelkan hubungan antara satu atau lebih variabel
independen (biasanya dilambangkan sebagai (X) dan variabel dependen
(Y). Kurva regresi ini mencoba untuk menunjukkan pola atau tren dalam
data dan memungkinkan kita untuk membuat prediksi atau estimasi
berdasarkan model tersebut.\\
Secara umum, terdapat dua jenis kurva regresi yang umum digunakan
yaitu regresi linier dan regresi non-linier.

Regresi Linier adalah garis lurus yang digunakan untuk memodelkan
hubungan antara variabel independen (X) dan variabel dependen (Y).

Persamaan regresi linier umumnya ditulis sebagai\\
\end{eulercomment}
\begin{eulerformula}
\[
Y=mX+b
\]
\end{eulerformula}
\begin{eulercomment}
di mana m adalah kemiringan (slope) dan b adalah perpotongan sumbu-y
(intercept).

Regresi linier dapat dilakukan dengan fungsi polyfit () atau berbagai
fungsi fit.

Sebagai permulaan kita menemukan garis regresi untuk data univariat
dengan polyfit(x, y, 1).

\end{eulercomment}
\eulersubheading{Berikut contoh menggambar kurva regresi di EMT}
\begin{eulercomment}
Contoh 1
\end{eulercomment}
\begin{eulerprompt}
>x=1:10; y=[2,3,1,5,6,3,7,8,9,8]; writetable(x'|y',labc=["x","y"])
\end{eulerprompt}
\begin{euleroutput}
           x         y
           1         2
           2         3
           3         1
           4         5
           5         6
           6         3
           7         7
           8         8
           9         9
          10         8
\end{euleroutput}
\begin{eulerprompt}
>p=polyfit(x,y,1)
\end{eulerprompt}
\begin{euleroutput}
  [0.733333,  0.812121]
\end{euleroutput}
\begin{eulerprompt}
>w &= "exp(-(x-10)^2/10)"; pw=polyfit(x,y,1,w=w(x))
\end{eulerprompt}
\begin{euleroutput}
  [4.71566,  0.38319]
\end{euleroutput}
\begin{eulerprompt}
>figure(2,1); ...
>figure(1); statplot(x,y,"p",xl="Regression"); ...
>  plot2d("evalpoly(x,p)",>add,color=blue,style="--"); ...
>  plot2d("evalpoly(x,pw)",5,10,>add,color=red,style="--"); ...
>figure(2); plot2d(w,1,10,>filled,style="/",fillcolor=red,xl=w); ...
>figure(0):
\end{eulerprompt}
\eulerimg{27}{images/EMTStatistika_SUB6_22305141013_Sherlyta Icha-012.png}
\eulerheading{Kurva Fungsi Kerapatan Probabilitas}
\begin{eulercomment}
Fungsi kerapatan/kepadatan probabilitas adalah fungsi yang memberikan
kemungkinan bahwa nilai suatu variabel acak akan berada di antara
rentang nilai tertentu.

Grafik fungsi kepadatan probabilitas berbentuk kurva lonceng. Area
yang terletak di antara dua nilai tertentu memberikan probabilitas
hasil observasi yang ditentukan.

Berikut contoh kurva fungsi kepadatan probabolitas.
\end{eulercomment}
\begin{eulerprompt}
>plot2d("qnormal(x,1,1.5)",-4,6);
>plot2d("qnormal(x,1,1.5)",a=2,b=5,>add,>filled):
\end{eulerprompt}
\eulerimg{27}{images/EMTStatistika_SUB6_22305141013_Sherlyta Icha-013.png}
\begin{eulercomment}
Probabilitas variabel acak x yang terletak antara 2 dan 5 memenuhi\\
P(2\textless{}X\textless{}5)= luas daerah hijau
\end{eulercomment}
\eulerheading{Kurva Distribusi Kumulatif}
\begin{eulercomment}
Kurva distribusi kumulatif (CDF) adalah representasi kumulatif dari
fungsi distribusi probabilitas suatu variabel acak. CDF memberikan
probabilitas bahwa variabel acak tersebut kurang dari atau sama dengan
suatu nilai tertentu.

Seringkali, grafik CDF disajikan dalam bentuk kurva monotonik yang
terus meningkat, dan ini memberikan gambaran visual yang baik tentang
distribusi probabilitas variabel acak.

\end{eulercomment}
\eulersubheading{Berikut menggambar kurva distribusi kumulatif di EMT}
\begin{eulercomment}
Contoh 1
\end{eulercomment}
\begin{eulerprompt}
>plot2d("normaldis",-4,4): 
\end{eulerprompt}
\eulerimg{27}{images/EMTStatistika_SUB6_22305141013_Sherlyta Icha-014.png}
\begin{eulercomment}
Dapat kita lihat dalam kurva CDF kontinu di atas dibagi menjadi 3
bagian, yaitu:\\
1. Bernilai 0 untuk x kurang dari batas bawah daerah rentang.\\
2. Merupakan fungsi monoton naik pada daerah rentang.\\
3. Bernilai konstan 1 untuk x lebih dari batas atas daerah rentangnya.

Contoh 2\\
Diberikan variabel acak dengan PDF sebagai berikut:

\end{eulercomment}
\begin{eulerformula}
\[
f(x) = \begin{cases} \frac{6}{5}(x^2+x) & 0\le x\le 1 \\ 0 & \text{x yang lain}. \end{cases}
\]
\end{eulerformula}
\begin{eulercomment}
Untuk menggambar grafik CDF-nya, pertama kita cari terlebih dahulu CDF
dari fungsi tersebut.

Untuk x pada interval\\
\end{eulercomment}
\begin{eulerformula}
\[
(- \infty,x)
\]
\end{eulerformula}
\begin{eulerformula}
\[
F(x)= \int_{- \infty}^{x} f(x)\ dx
\]
\end{eulerformula}
\begin{eulerformula}
\[
F(x)= \int_{- \infty}^{x} 0\ dx
\]
\end{eulerformula}
\begin{eulerformula}
\[
F(x)= 0
\]
\end{eulerformula}
\begin{eulercomment}
Untuk x pada interval [0,1]\\
\end{eulercomment}
\begin{eulerformula}
\[
F(x)= \int_{- \infty}^{x} f(x)\ dx
\]
\end{eulerformula}
\begin{eulerformula}
\[
F(x)= \int_{- \infty}^{0} f(x)\ dx + \int_{0}^{x} f(x)\ dx
\]
\end{eulerformula}
\begin{eulerformula}
\[
F(x)= \int_{- \infty}^{0} 0\ dx + \int_{0}^{x} \frac{6}{5} (x^2+x)\ dx
\]
\end{eulerformula}
\begin{eulerformula}
\[
F(x) = 0 + \frac{6}{5}(\frac{x^3}{3} + \frac{x^2}{2})
\]
\end{eulerformula}
\begin{eulerformula}
\[
F(x) = \frac{6}{5}(\frac{x^3}{3} + \frac{x^2}{2})
\]
\end{eulerformula}
\begin{eulercomment}
Untuk x pada interval\\
\end{eulercomment}
\begin{eulerformula}
\[
(1, \infty)
\]
\end{eulerformula}
\begin{eulerformula}
\[
F(x)= \int_{- \infty}^{x} f(x)\ dx
\]
\end{eulerformula}
\begin{eulerformula}
\[
F(x)= \int_{- \infty}^{0} f(x)\ dx + \int_{0}^{1} f(x)\ dx + \int_{1}^{x} f(x)\ dx
\]
\end{eulerformula}
\begin{eulerformula}
\[
F(x)= \int_{- \infty}^{0} 0\ dx + \int_{0}^{1} \frac{6}{5} (x^2+x)\ dx +  \int_{1}^{\infty} 0\ dx
\]
\end{eulerformula}
\begin{eulerformula}
\[
F(x) = 0 + \frac{6}{5} (\frac{1^3}{3} + \frac{1^2}{2}) + 0
\]
\end{eulerformula}
\begin{eulerformula}
\[
F(x) = \frac{6}{5} \frac{5}{6} = 1
\]
\end{eulerformula}
\begin{eulercomment}
Sehingga, diperoleh CDF :\\
\end{eulercomment}
\begin{eulerformula}
\[
F(x) = \begin{cases} 0 & x<0 \\ \frac{6}{5}(\frac{x^3}{3} + \frac{x^2}{2}) & 0\le x\le 1 \\ 1 & x>1. \end{cases}
\]
\end{eulerformula}
\begin{eulercomment}
Selanjutnya, akan kita gambarkan grafik CDF tersebut di EMT.\\
Pertama, kita definisikan terlebih dahulu fungsi f(x) dengan
menggunakan fungsi function map.
\end{eulercomment}
\begin{eulerprompt}
>function map f(x) ...
\end{eulerprompt}
\begin{eulerudf}
    if x<0 then return 0
    else if x>=0 and x<=1 then return (6/5)*((x^3/3)+(x^2/2))
    else return 1
    endif;
  endfunction
\end{eulerudf}
\begin{eulercomment}
Kemudian, kita akan menggambar grafik fungsi di atas dengan
menggunakan fungsi plot2d() pada interval x dari -1 sampai 2.
\end{eulercomment}
\begin{eulerprompt}
>plot2d("f(x)",-1,2):
\end{eulerprompt}
\eulerimg{27}{images/EMTStatistika_SUB6_22305141013_Sherlyta Icha-032.png}
\end{eulernotebook}
\end{document}
